\documentclass{classes/report}

\usepackage{enumitem}
\pagestyle{empty}

\begin{document}

基礎電子工学 \ 第3回 \ 課題

\begin{flushright}
    \underline{\ 学籍番号 \ 2397002 \ 氏名 \ 杉本 謙仁 \ }
\end{flushright}

\bigskip

\begin{enumerate}
    \item ゾンマーフェルトのモデルにおいて,金属棒の長さを 3 [nm]として,n = 1 \sim \ 4 までのエネルギーを求めよ.
\end{enumerate}

ゾンマーフェルトのモデルにおけるエネルギーは次の式で表される.
\begin{equation}
    E_n = \frac{n^2\hbar^2}{8mL^2}
\end{equation}

ここで,$m$は電子の質量,$\hbar$はプランク定数,$L$は金属棒の長さである.

$n=1$の場合,エネルギーは次のように求められる.

\begin{equation}
    E_1 = \frac{(6.6 \times 10^{-34})^2}{8 \times 9.11 \times 10^{-31} \times (3 \times 10^{-9} )^2} = 6.6 \times 10^{-21} [J]
\end{equation}

$n=2$の場合,エネルギーは次のように求められる.

\begin{equation}
    E_2 = \frac{4 \times (6.6 \times 10^{-34})^2}{8 \times 9.11 \times 10^{-31} \times (3 \times 10^{-9} )^2} = 2.7 \times 10^{-20} [J]
\end{equation}

$n=3$の場合,エネルギーは次のように求められる.

\begin{equation}
    E_3 = \frac{9 \times (6.6 \times 10^{-34})^2}{8 \times 9.11 \times 10^{-31} \times (3 \times 10^{-9} )^2} = 6.0 \times 10^{-20} [J]
\end{equation}

$n=4$の場合,エネルギーは次のように求められる.

\begin{equation}
    E_4 = \frac{16 \times (6.6 \times 10^{-34})^2}{8 \times 9.11 \times 10^{-31} \times (3 \times 10^{-9} )^2} = 1.1 \times 10^{-19} [J]
\end{equation}

\bigskip

\begin{enumerate}[resume]
    \item 金属棒の長さを 10 [mm]としたとき,n = 1 と n = 2 との間のエネルギー差を求めよ.
\end{enumerate}

$n=1$の場合のエネルギーは次のように求められる.

\begin{equation}
    E_1 = \frac{(6.6 \times 10^{-34})^2}{8 \times 9.11 \times 10^{-31} \times (10 \times 10^{-3} )^2} = 6.0 \times 10^{-34} [J]
\end{equation}

$n=2$の場合のエネルギーは次のように求められる.

\begin{equation}
    E_2 = \frac{4 \times (6.6 \times 10^{-34})^2}{8 \times 9.11 \times 10^{-31} \times (10 \times 10^{-3} )^2} = 2.4 \times 10^{-33} [J]
\end{equation}

よって,n = 1 と n = 2 との間のエネルギー差は次のように求められる.

\begin{equation}
    E_2 - E_1 = 1.8 \times 10^{-33} [J]
\end{equation}


% \subfile{sections/title}
% \newpage

% \tableofcontents
% \clearpage

% \subfile{sections/abstract}
% \newpage

% \subfile{sections/main_issue}
% \newpage

% \subfile{sections/references}

\end{document}