\documentclass{classes/report}

\usepackage{enumitem}
\pagestyle{empty}

\begin{document}
% textlint-disable ja-technical-writing/ja-no-mixed-period
基礎電子工学 \ 第8回 \ 課題
% textlint-enable ja-technical-writing/ja-no-mixed-period
\begin{flushright}
    \underline{\ 学籍番号 \ 2397002 \ 氏名 \ 杉本 謙仁 \ }
\end{flushright}

\bigskip

\begin{enumerate}
    \item n 型半導体のドナー密度が $2 \times 1021 [\mathrm{m}^{-3}]$,
          p 型半導体のアクセプタ密度が $8 \times 1021 [\mathrm{m}^{-3}]$である
          シリコンの pn 接合ダイオードの $300 [\mathrm{K}]$における空乏層幅を求めよ.
          ただし,拡散電位を $0.7 [\mathrm{V}]$,シリコンの比誘電率を $12$ とする.
\end{enumerate}

空乏層幅 $\omega$ は,以下の式で求められる.
\begin{equation}
    \omega = \sqrt{\frac{2\epsilon_0 \epsilon_s V_{d} \left( N_{\mathrm{a}} + N_{\mathrm{d}} \right)}{q N_{\mathrm{a}} N_{\mathrm{d}}}}
\end{equation}

ここで,$N_{\mathrm{d}} = 2 \times 10^{21} [\mathrm{m}^{-3}]$,
$N_{\mathrm{a}} = 8 \times 10^{21} [\mathrm{m}^{-3}]$,
$V_{d} = 0.7 [\mathrm{V}]$,
$\epsilon_s = 12$,
$q = 1.6 \times 10^{-19} [\mathrm{C}]$,
$\epsilon_0 = 8.85 \times 10^{-12} [\mathrm{F/m}]$とすると,空乏層幅は以下のように求められる.

\begin{equation}
    \begin{split}
        \omega &= \sqrt{\frac{2 \times 8.85 \times 10^{-12} \times 12 \times 0.7 \times \left( 2 \times 10^{21} + 8 \times 10^{21} \right)}{1.6 \times 10^{-19} \times 2 \times 10^{21} \times 8 \times 10^{21}}} \\
        &= 7.6 \times 10^{-7} [\mathrm{m}]
    \end{split}
\end{equation}

\bigskip

\begin{enumerate}[resume]
    \item 接合面積を $2 \times 10^{-2} [\mathrm{mm}^2]$として,前述の課題の接合容量を求めよ.
\end{enumerate}

接合容量 $C_{j}$ は,以下の式で求められる.
\begin{equation}
    C_{j} = \frac{\epsilon_0 \epsilon_s A}{\omega}
\end{equation}

ここで,$A = 2 \times 10^{-2} [\mathrm{mm}^2] = 2 \times 10^{-8} [\mathrm{m}^2]$とすると,接合容量は以下のように求められる.

\begin{equation}
    \begin{split}
        C_{j} &= \frac{8.854 \times 10^{-12} \times 12 \times 2 \times 10^{-8}}{7.62 \times 10^{-7}} \\
        &= 2.8 \times 10^{-12} [\mathrm{F}] \\
        &= 2.8 [\mathrm{pF}]
    \end{split}
\end{equation}

\end{document}