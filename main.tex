\documentclass{classes/report}

\usepackage{enumitem}
\pagestyle{empty}

\begin{document}
% textlint-disable ja-technical-writing/ja-no-mixed-period
基礎電子工学 \ 第9回 \ 課題
% textlint-enable ja-technical-writing/ja-no-mixed-period
\begin{flushright}
    \underline{\ 学籍番号 \ 2397002 \ 氏名 \ 杉本 謙仁 \ }
\end{flushright}

\bigskip

\begin{enumerate}
    \item 光を照射して,過剰少数キャリアである電子を $6 \times 10^{17} [\mathrm{cm}^{-3}]$発生させた.電子の寿命を $20 [\mathrm{ms}]$として,$100 [\mathrm{ms}]$後の電子密度を求めよ.
\end{enumerate}

\begin{equation}
    \Delta n(t) = 6 \times 10^{17} = G_{L} \tau_{n}
\end{equation}

より,電子密度 $\Delta n(t)$ は以下の通り求められる.

\begin{equation}
    \begin{split}
        \Delta n(t) &= \Delta n_0 \exp\left(-\frac{t}{\tau}\right) \\
        &= 6 \times 10^{17} \times \exp\left(-\frac{100 \times 10^{-6}}{20 \times 10^{-6}}\right) \\
        &= 4.0 \times 10^{15} [\mathrm{cm}^{-3}]
    \end{split}
\end{equation}

\bigskip

\begin{enumerate}[resume]
    \item 室温($300 [\mathrm{K}]$)において n 型半導体の電子の移動度が $4,000[cm^{2}/\mathrm{Vs}]$とすると,電子の拡散係数を求めよ.
\end{enumerate}

アインシュタインの関係式より,電子の拡散係数 $D_{n}$ は以下の通り求められる.

\begin{equation}
    D_{n} = \frac{kT}{q}\mu_{n} = \frac{1.38 \times 10^{-23} \times 300}{1.6 \times 10^{-19}}\times 4000 \times 10^{-4} = 0.01 [\mathrm{m}^{2}/\mathrm{s}]
\end{equation}

\end{document}