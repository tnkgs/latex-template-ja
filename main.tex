\documentclass{classes/report}

\usepackage{enumitem}
\pagestyle{empty}

\begin{document}
% textlint-disable ja-technical-writing/ja-no-mixed-period
基礎電子工学 \ 第4回 \ 課題
% textlint-enable ja-technical-writing/ja-no-mixed-period
\begin{flushright}
    \underline{\ 学籍番号 \ 2397002 \ 氏名 \ 杉本 謙仁 \ }
\end{flushright}

\bigskip

\begin{enumerate}
    \item シリコンに添加されたドナー不純物は,
          室温付近では自由電子を放出して正に帯電し,放出した自由電子は,
          正に帯電したドナーとの間のクーロン力によって束縛されてドナーのまわりを回っている.
          電子軌道を円軌道,シリコンの比誘電率を 12 として,その軌道半径を求めよ.
\end{enumerate}

ドナー不純物によって放出された自由電子は,ドナー不純物の正に帯電したイオンとクーロン力によって束縛され,
ドナー不純物の周りを円軌道で回っている.
このとき,ボーアのモデルにより,電子の運動エネルギーとクーロン力の大きさが等しくなる.
軌道半径の式は以下の通りである.

\begin{equation}
    r = \frac{\epsilon \hbar^2}{\pi mq^2}n^2
\end{equation}

ここで,$m$は自由電子の質量,$r$は軌道半径$\hbar$はプランク定数,$\epsilon$はシリコンの誘電率,$q$は電子の電荷量を表す.
また,シリコンの比誘電率は12のため,$\epsilon = 12\epsilon_0$となる.$\epsilon_0$は真空の誘電率である.
そして,$n$は軌道の番号であり,$n=1$を利用する.
これらを代入すると以下の通りある.

\begin{equation}
    r = \frac{12\times 8.854 \times 10^{-12} \times \left(6.6\times 10^{-34} \right)^2}{3.14 \times 9.11 \times 10^{-31} \times \left(1.6\times 10^{-19} \right)^2} \approx 6.3\times 10^{-10} [\mathrm{m}]
\end{equation}

\bigskip

\begin{enumerate}[resume]
    \item シリコン中のドナーから電子を奪って自由電子にするのに必要なエネルギーを求めよ.
          ただし,シリコンの比誘電率は 12 とする.
\end{enumerate}

$n=\infty$のエネルギーと$n=1$のエネルギーの差を求めることで,ドナー不純物から電子を奪って自由電子にするのに必要なエネルギーを求めることができる.

\begin{equation}
    \begin{split}
        E_\infty - E_1 &= - \frac{mq^4}{8\epsilon^2\hbar^2}\left(\frac{1}{\infty^2} - \frac{1}{1^2}\right) \\
        &= \frac{9.11\times 10^{-31} \times \left(1.6 \times 10^{-19} \right)^3}{8 \times \left(12 \times 8.85 \times 10^{-12} \right)^2 \times \left(6.6 \times 10^{-34} \right)^2} \\
        &\approx 0.094 [\mathrm{eV}]
    \end{split}
\end{equation}

\end{document}