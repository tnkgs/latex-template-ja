\documentclass{classes/report}

\usepackage{enumitem}
\pagestyle{empty}

\begin{document}
% textlint-disable ja-technical-writing/ja-no-mixed-period
基礎電子工学 \ 第12回 \ 出席
% textlint-enable ja-technical-writing/ja-no-mixed-period
\begin{flushright}
    \underline{\ 学籍番号 \ 2397002 \ 氏名 \ 杉本 謙仁 \ }
\end{flushright}

\bigskip

\begin{enumerate}
    \item 電子密度 $3 \times 10^{19} [\mathrm{m}^{-3}]$のn型シリコン半導体に直径$0.5[\mathrm{mm}]$のアルミニウム電極
          によるショットキー接合をつくった.シリコンの比誘電率を12,$Vd = 0.75 [\mathrm{V}]$として,無バイアス時の接合容量を求めよ.
\end{enumerate}

\begin{equation}
    \begin{split}
        \omega & = \sqrt{\frac{2\epsilon_0 \epsilon_s}{qN_d}} \sqrt{V_d - V}                                                             \\
               & = \sqrt{\frac{2 \times 8.854 \times 10^{-12} \times 12}{1.602 \times 10^{-19} \times 3 \times 10^{19}}} \sqrt{0.75 - 0} \\
               & = 5.8 \times 10^{-6} [\mathrm{m}]                                                                                       \\
        C      & = \frac{\epsilon}{\omega} \cdot S                                                                                       \\
               & = \frac{8.854 \times 10^{-12} \times 12}{5.8 \times 10^{-6}} \times 3.14 \times \left(0.25 \times 10^{-3} \right)^2     \\
               & = 3.6 \times 10^{-12}                                                                                                   \\
               & = 3.6 [\mathrm{pF}]
    \end{split}
\end{equation}

\end{document}